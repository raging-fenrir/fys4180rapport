\section {Introduction}
  The Mach-Zendher interferometer was first proposed in 1891 by Ludwig Zendher
  \cite{lz} and later on refined by Ludwig Mach\cite{lm}.
  In the later years it has been used for including understanding the kinetics of dense
  fluid transport properties\cite{th,dym} and diffusion measurements \cite{dag}.
  Within high pressure experiments a classical showcase of agreement between
  experimental data and theory is the selfdiffusion data of methane\cite{ht,ew}.
  Exactly this interferometer were used by D. K. Dysthe in 1995 to measure interdiffusion
  in binary liquid mixtures, NaCl/water and 1-butanol/water. It was then found to 
  have a precision of 0.6\% for NaCl and 1.4\% for 1-butanol\cite{dag}.

  We want to find out how accurate our setup is and what might be
  done with it to increase this accuracy. There is also the question of how well
  the phase-unwrapping tools available can handle these high-precision measurements.
  With this information the Mach-Zendher setup can be more permenantly constructed 
  and be used for future experiments.
  
