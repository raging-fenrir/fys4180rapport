\section {Result}
\subsection {Test system}
The diffusion of table salt in water gave the expected results from a diffusion
experiment. As seen in \autoref {diff1}, \autoref {diff2}, \autoref {diff3} and 
\autoref {diff4} the unwrapping picture shows a clear inclination which means that
the optical path length has been increased and the difference between a pahse shift
of the background to the next measurement has increased. The offset behaviour is
due to too much salt being used and gave off some optical disturbance. From 
\autoref {diff1} to \autoref {diff4} the phase picture has behaved as expected.
The presicion seem to be adequate in that the phase unwrapping picks up on the
changes in optical path length, the piezo control moves the phase at increments
small enough, and the calibration of $1.12\cdot V$ is giving a correct phase shift.

\subsection {Calcit erosion}
\subsubsection {By water}
There were no change in the phase picture and no visible erosion in the raw camera
feed. A possible reason for this was the water pressure, it may have been too low.
If we had more time and could wait a day or two more, there should have been som change.

\subsubsection {By water with salt peter acid}
With the added salt peter acid and same rate of flow, 10 nl/s, the erosion started 
after 1-2 hours, and became visible on the raw feed from the camera. There were no
distinct features in the phase picture, as seen in \autoref {cal1}, \autoref {cal2},
\autoref {cal3} and \autoref {cal4}. Though there is noe data to be collected from
these pictures they give information about what the next step should be.


